% Please do not change the document class
\documentclass{scrartcl}

% Please do not change these packages
\usepackage[hidelinks]{hyperref}
\usepackage[none]{hyphenat}
\usepackage{setspace}
\doublespace

% You may add additional packages here
\usepackage{amsmath}
\usepackage{graphicx}
\usepackage{wrapfig}
\graphicspath{ {./images/} }

% Please include a clear, concise, and descriptive title
\title{Comp140 Usability Analysis} 

% Please do not change the subtitle
\subtitle{Comp140 Usability Analysis}

% Please put your student number in the author field
\author{1507516}

\begin{document}

\maketitle

\abstract{}

\section{Game controller and Evaluation}

The game controller was a minimalist design because the controls for the game were fairly simple. It featured directional buttons, and a trigger button at the end which made the player attack. The controller went through a few iterations before coming to it's final design. The initial design was made out of playdough and had a lot of issues due to the playdough being conductive. For the second sprint the controller was re-designed out of cardboard and paper, with electrical paint for the directional arrows, and playdough for the trigger button.

\section{Recommended improvements}

Improvement one:
the controller requires more ergonomic design as the design was too hard to hold and press the button at the end. This could of been done by reducing the length of the contorller so that players with small hands could reach the end button easily.

Improvement two:
Another button was needed for the block mechanic, although with the version of the game I was working with did not have a working block merchanic, this would of been needed for the final version of the game where it did work.


\section{Conclusion}

\bibliographystyle{ieeetr}
\bibliography{comp140_Usability}

\end{document}
